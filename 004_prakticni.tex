Praktični dio napravljen je temeljem rada \citep{resnik2010gibbs}. Resnik i Hardisty objašnjavaju ostvarenje Naivnog Bayesa \engl{Naive Bayes} pomoću Gibbsovog uzorkovanja na primjeru klasifikacije polariteta dokumenata prema riječima u dokumentima. 

U praktičnom dijelu napravit će se sustav koji izvodi algoritam Naivnog Bayesa, a potrebne vjerojatnosti računa Gibbsovim uzorkovanjem. 

\section{Skup podataka}

Besplatno dostupna biblioteka \textit{Natural Language Toolkit, NLTK} \footnote{Dostupno na \url{http://www.nltk.org/} .} za programski jezik \textit{Python} sadrži pripremljene korpuse teksta. 
Za potrebe Gibbsovog uzorkovanja korišten je skup podataka \textit{movie\_reviews} koji sadrži osvrte na filmove. Format zapisa u korpusu je $<T, S>$, gdje je $T$ tekst osvrta, a $S \in \{'pos', 'neg'\}$ označeni polaritet osvrta. U nastavku će se za \textit{'pos'} kritike koristiti broj $1$, a za \textit{'neg'} kritike broj $0$. Pozitivne kritike označene su oznakom \textit{'pos'}, a negative \textit{'neg'}. Primjer negativne kritike prikazan je unutar slike \ref{fig:critic_example}.

Programski jezik korišten za \textit{Python}, verzija 2.7.3, u 64-bitnom okruženju. Dodatne \textit{Python} biblioteke korištene prilikom izrade programa su \textit{numpy}\footnote{Dostupno na \url{http://www.numpy.org/}.} i već spomenuti \textit{nltk}.

\begin{figure}
\begin{bclogo}
{Universal soldier} \footnotesize{ ex - universal soldier luc has to battle a group of newer - model engineered fighters gone bad . the review jean - claude van damme has a one - liner early on in universal soldier : the return , his latest attempt to remain relevant , that sums up this entire movie ; he says " been there , done that . " no film critic could possibly sum up van damme ' s recent film choices any better . while other ageing action stars have wisely moved into other film genres ( schwarzenegger makes as many family comedies as he does action films ) , van damme stubbornly persists in sticking with what used to work for him : martial arts and guns . this unwillingness or perhaps inability to move into new genres has caused van damme to enter the straight to video world , with legionnaire never seeing the inside of a multiplex . he joins fellow martial artist / action star steven seagal as they watch their film careers rapidly fizzle away . universal soldier : the return is truly poor . the plot is a complete copy of several action films from this decade , specifically terminator 2 : judgement day and the similarly named soldier . soldier ' s kurt russell was an older model super - soldier sent off to retirement when circumstances forced him to battle his successors , for the good of a planet ; schwarzenegger ' s terminator in t2 tried to save john connor from a newer model killing machine , the t - 1000 ; and jean - claude , a former universal soldier , has to save the planet from the rampage of a group of , you guessed it , newer model soldiers .}
\end{bclogo}
\caption{Primjer negativnog osvrta iz movie\_reviews baze podataka}
\label{fig:critic_example}
\end{figure}

\section{Matematički model}

U ovom slučaju, dokument je skup riječi koje sadrži, tzv. \engl{bag of words} princip. Za dokument $W_j$ potrebno je dodijeliti adekvatan polaritet $L_j = 0$ ili $L_j = 1$. Skup dokumenata $\mathbb{C}_{k}$ pripada skupini klase $L_j = k$, a dobije se tako da se prebroje svi dokumenti $W_j$ s $L_j=k$, prema tome $\mathbb{C}_{k} = \{W_j | L_j = k\}$. Potrebno je pronaći polaritet $L_j$ koji, za poznati dokument $W_j$, pronalazi maksimalnu vjerojatnost $P(L_j|W_J)$. Prema Bayesovom pravilu vrijedi:
\begin{equation}
L_j = \argmax_L P(L|W_j) = \argmax_L \frac{P(W_j|L)P(L)}{P(W_j)}.
\end{equation}
Moguće je izostaviti nazivnik $P(W_j)$ jer nije ovisan o $L_j$. Na ovaj način nastoji se modelirati način na koji su dokumenti nastali, što se naziva generativnim modelom \engl{Generative model}. Odabir polariteta $L_j$ modelira se Bernoulijevom raspodjelom s parametrom $\pi$:
\begin{equation}
L_j \sim Bernoulli(\pi),
\end{equation} Potrebno za svaku poziciju riječi u dokumentu $R_i$ odabrati riječ $w_j$ temeljem distribucije vjerojatnosti riječi. Odabir distribucije vjerojatnosti iz koje se uzorkuje ovisan je dodijeljenom polaritetu dokumenta $L_j$. Moguće distribucije označavat će se $\theta_0$ i $\theta_1$. Dokument $W_j$ gradit će se temeljem multinomijalne distribucije:
\begin{equation}
W_j = Multinomijalna(R_j, \theta_{L_{j}}).
\end{equation}

Pretpostavlja se da je uzorkovanje međusobno neovisno. Distribucijama $L_j$ i $W_j$ nastoji se aproksimirati način na koji su dobiveni stvarni podaci. 

\subsection{Apriori parametri}

Gore spomenute parametre distribucija $\pi$ i $\theta$ nužno je imati prije generiranja raspodjela za $W_j$ i $L_j$. Dobivanje početnih vrijednosti za $\pi$ i $\theta$ će se dobiti iz jednolike raspodjele. Konkretno, $\pi$ će se generirati Beta distribucijom s parametrima $\gamma_{\pi 1} = 1$ i $\gamma_{\pi 2} = 1$:
\begin{equation}
\pi \sim Beta(\gamma_{\pi}).
\end{equation}
Parametri apriori vrijednosti nazivaju se hiperparametrima. Kako oba parametra Beta distribucije ovdje iznose 1, svi događaji su jednako vjerojatna, što znači da je apriori znanje o sustavu nedostupno. U slučaju $\theta$ parametra, on je modeliran Dirichlechtovom distribucijom, generaliziranom Beta distribucijom važećoj u više od dvije dimenzije:
\begin{equation}
\theta \sim Dirichlet(\gamma_{\theta})
\end{equation}

