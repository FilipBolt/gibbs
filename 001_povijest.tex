Stuart i Donald Geman su prvi puta opisali postupak Gibbsovog uzorkovanja \citep{geman1984stochastic}. Braća Geman bavili su se izradom modela za analizu slike. Gibbsovo uzorkovanje u njihovom radu bio je poseban slučaj Metropolis-Hastings algoritma \citep{metropolis1953equation}. \citep{gelfand1990sampling} su pokazali potencijalne primjene Gibbsovog uzorkovanja prilikom rješavanja velikog broja statističkih problema.

Metoda Gibbsovog uzorkovanja kasnije se koristila za uzorkovanje skupova podataka s velikim brojem varijabli. Gibbsovo uzorkovanje najčešće se koristi kada su zadovoljena dva preduvjeta. Prvi preduvjet nalaže da zajednička distribucija \engl{joint distribution} nije eksplicitno poznata ili je zahtjevno izravno uzorkovati iz zajedničke distribucije. Drugi preduvjet je poznata uvjetna distribucija svake varijable te mogućnost relativno jednostavnog uzorkovanja iz uvjetnih distribucija. 

Gibbsov algoritam uzorkovanja za svaku varijablu generira nizove iz pripadajućih uvjetnih distribucija.  